\chapter{Экспериментальная часть}
\label{cha:research}

\section{Пример работы}

Пример работы программы приведен на рисунке \ref{img:prog_example}. На вход алгоритма был подан файл, содержащий строку ".BANANA.".

\imgsc{h}{0.9}{prog_example}{Пример работы программы}

Результат работы программы можно посмотреть, например, с помощью программы ghex \cite{r4} (рисунок \ref{img:result}).

\imgsc{h}{0.75}{result}{Результат работы программы}

На рисунке \ref{img:result} видно, что размер полученного алфавита 4, алфавит состоит из символов ".ABN",, а последовательность символов выглядит следующим образом: [1, 1, 3, 0, 3, 2, 3, 0];

\section{Описание лабораторного оборудования}

Технические характеристики устройства, на котором выполнялосьтестирование:
\begin{itemize}
	\item наименование дистрибутива операционной системы: Linux Mint 19.3 Tricia;
	\item объем оперативной памяти: 8 Гб;
 	\item наименование модели процессора: Intel(R) Core(TM) i5-8250U CPU @ 1.60ГГц \cite{r5};
 	\item количество физических ядер процессора: 4;
 	\item количество логических ядер процессора: 8.
\end{itemize}

Тестирование производилось на ноутбуке, подключенном к сети электропитания.

\section{Сбор экспериментальных данных}

// Написать про замеры

В таблице \ref{tab_1} приведены результы замеров времени работы для реализации конвейеризированного алгоритма.

\begin{table}[H]
	\begin{center}\small
		\caption{Замеры времени (в мкс.), часть 1}
		\label{tab_1}
		""\newline
		\begin{tabular}{|c|c|c|c|c|c|c|c|c|}
		\hline
		\specialcell{Кол. \\ файлов} & \specialcell{Номер \\ конвейера} & \specialcell{СВО} & МинОж & СОж & МаксОж & МинОб & СОб & МаксОб	\\
		\hline
		\hline
		\multirow{4}*{60} & 1 & \multirow{4}*{173.7} & 1 & 6.73 & 82 & 11 & 19.82 & 267 \\
		\cline{2-2} & 2 &  & 4 & 6.52 & 98 & 27 & 33.1 & 164  \\
		\cline{2-2} & 3 &  & 5 & 8.75 & 111 & 19 & 27.03 & 178  \\
		\cline{2-2} & 4 &  & 4 & 8.9 & 124 & 45 & 62.9 & 360  \\
		\hline
		\multirow{4}*{80} & 1 & \multirow{4}*{165.3} & 1 & 5.925 & 32 & 10 & 18.21 & 185 \\
		\cline{2-2} & 2 &  & 3 & 6.76 & 101 & 22 & 27.34 & 150  \\
		\cline{2-2} & 3 &  & 2 & 8.73 & 66 & 18 & 24.58 & 90  \\
		\cline{2-2} & 4 &  & 4 & 6.86 & 91 & 50 & 66.9 & 335 \\
		\hline
		\multirow{4}*{100} & 1 & \multirow{4}*{152.2} & 4 & 4.74 & 29 & 11 & 15.83 & 222 \\
		\cline{2-2} & 2 &  & 3 & 6.69  & 180 & 21 & 26.39 & 137  \\
		\cline{2-2} & 3 &  & 3 & 6.95  & 93 & 18 & 22.51 & 120  \\
		\cline{2-2} & 4 &  & 4 & 6.68  & 104 & 44 & 62.4 & 327  \\
		\hline
		\end{tabular}
	\end{center}
\end{table}

Примечание:
\begin{itemize}
	\item СВО --- среднее время обработки заявки;
	\item МинОж --- минимальное время ожидания;
	\item СОж --- среднее время ожидания;
	\item МаксОж --- максимальное время ожидания;
	\item МинОб --- минимальное время обработки;
	\item СОб --- среднее время обработки;
	\item МаксОб --- максимальное время обработки.
\end{itemize}

В таблице \ref{tab_2} приведены результы замеров времени работы для реализации последовательного алгоритма.

\begin{table}[H]
	\begin{center}\small
		\caption{Замеры времени (в мкс.), часть 1}
		\label{tab_2}
		""\newline
		\begin{tabular}{|c|c|}
		\hline
		\specialcell{Количество файлов} & \specialcell{СВО}	\\
		\hline
		\hline
		60 & 233.183 \\
		\hline
		80 & 208.762 \\
		\hline
		100 & 216.83 \\
		\hline
		\end{tabular}
	\end{center}
\end{table}

\section{Анализ экспериментальных данных}

На рисунках \ref{img:output_1} - \ref{img:output_2} приведены графики времени работы реализаций алгоритмов на основе таблиц \ref{tab_3} и \ref{tab_4}.

%\imgsc{H}{0.8}{output_1}{Зависимость времени работы алгоритма сортировки слияния от размера массива}
%\imgsc{H}{0.8}{output_2}{Зависимость времени работы алгоритма сортировки вставками от размера массива}

На основании полученных данных можно сделать некоторые выводы.
\begin{enumerate}
	\item При задействовании четырех потоков и использовании параллельного алгоритма можно сократить время сортировки слиянием в 2.25 раза (при размере массива более 5000 элементов).
	\item Задействование иного числа потоков или использование реализации последовательного алгоритма приводит к увеличению времени выполнения упорядочивания массива.
	\item Разница между временем работы параллельной реализации при задействовании 1 рабочего потока и временем работы реализации последовательного алгоритма является постоянной и равняется в среднем 0.32 мс. 
	\item Неиспользование основного потока для упорядочивания части исходного массива привело к тому, что общее время выполнения сортировки при задействовании 8 потоков больше, чем время выполнения сортировки при задействовании 4 потоков.
\end{enumerate}

\section{Вывод из экспериментальной части}

Таким образом, можно заключить, что для решения задачи сортировки слиянием\footnote{При условии, что сортировка будет выполняться по достаточно сложному ключу} оправдано задействование нескольких потоков при работе на многопроцессорной системе. При этом их число не должно превышать число логических ядер процессора, так как в ином случае потоки будут конкурировать друг с другом, увеличивая при этом общее время работы программы.
