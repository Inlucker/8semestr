\chapter{Конструкторский раздел}
\label{cha:design}

\section{Разработка алгоритма}
Рассмотренный алгоритм GSP в основном используется в торговли, и обычно под транзакцией имеется ввиду покупка определенного набора товаров (продуктов, книг и т.д.). В случае анализа активности пользователей САПР, зачастую, данные представлены в виде последовательности команд с параметрами, выполненных в определенный момент времени. Кроме этого, данный алгоритм учитывает кто совершил покупку, то есть id клиента.
Поэтому за одну транзакцию примем выполнение одной команды, а за id клиента - id сессии. В таком случае поддержкой последовательности команд будет отношение числа сессий поддерживающих данную последовательность к общему их количеству. 


