\StructuredChapter{РЕФЕРАТ}

Расчетно-пояснительная записка содержит 32 с., 1 табл., 16 источников.

Ключевые слова: анализ пользовательской активности, последовательные шаблоны, математическая модель активности пользователей ПО, ассоциативные правила.

%Объектом научно-исследовательской работы является пользовательская активность.

%\textbf{Цель работы} – провести обзор существующих методов анализа пользовательской активности, сформулировать критерии для их оценки и провести сравнение рассмотренных методов.
%
%\textbf{Задачи работы:}
%\begin{itemize}
%	\item рассмотреть существующие решения в области анализа пользовательской активности;
%	\item классифицировать методы анализа пользовательской активности;
%	\item выбрать для них критерии оценки и сравнить.
%\end{itemize}

В результате выполнения научно-исследовательской работы был представлен обзор разных подходов к анализу пользовательской активности, рассмотрены и классифицированы существующие методы, а также проведено их сравнение.